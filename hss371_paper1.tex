\documentclass[12pt]{article}
\usepackage[letterpaper]{geometry}
\geometry{top=1.0in, bottom=1.0in, left=1.0in, right=1.0in}
\usepackage{setspace}
\doublespacing
\usepackage{times}
\usepackage{fancyhdr}
\pagestyle{fancy}
\lhead{}
\chead{}
\rhead{Saleem \thepage}
\lfoot{}
\cfoot{}
\rfoot{}
\renewcommand{\headrulewidth}{0pt}
\renewcommand{\footrulewidth}{0pt}
\setlength\headsep{0.333in}
\newcommand{\bibent}{\noindent \hangindent 40pt}
\newenvironment{workscited}{\newpage \begin{center} Works Cited \end{center}}
{\newpage}
\begin{document}
\begin{flushleft}
    Khayyam Saleem\\
    Professor Michael Kowal\\
    HSS 371EV -- Computers and Society\\
    15 October 2017\\
\begin{center}Privacy vs Freedom\end{center}
    \setlength{\parindent}{0.5in}
    %BODY OF PAPER
    \par The liberties of the American citizen have always been a topic of
    deliberation and enumeration. Formally yet abstractly listed in the
    Constitution, we have certain ``inalienable rights," a reward we reap in
    return for compliance with the government of and maintaining membership in
    the United States of America. Among these rights, we have some that are
    concrete and relatively unambiguous: the freedom of speech, the right to 
    bear arms, right to a speedy and public trial, etc. However, we also have 
    quite a few unenumerated or implied rights. Since the Constitution and the 
    Bill of Rights are so open to interpretation, the unenumerated rights we 
    have are often contested. One of the most controversial, as of late, has 
    been the right to privacy. In general, we tend to agree that we DO have the 
    right to privacy, provided that the maintenance and protection of an 
    entity's privacy does not infringe upon the maintenance and protection of 
    national security or the rightful upholding and enforcement of the law. This 
    boundary, however, is a difficult one to assess; so difficult that it almost 
    necessitates the questions: in what scenarios can government compromise 
    individual privacy for the sake of national security, what governing body 
    gets to make that decision, and what can we do to ensure that this is not 
    being abused? As important as national security is, we absolutely must 
    respect the implied rights to privacy that the Constitution allows us, lest 
    we face the repercussions of a dystopian society imagined (or perhaps 
    foretold) by the likes of George Orwell or Ayn Rand.

    \par The Constitution and the Bill of Rights is the primary source of our 
    implied right to privacy and information security as American citizens.  
    Given that we have trusted, protected, and honored this document for as long 
    as our country has lived, we should understand that it is a living, 
    breathing document, whose interpretations and stipulations will vary 
    dependent on the temporal and societal context. In and around 1792, the 
    representation of privacy that could be afforded to people was that they 
    should be able to enjoy the sanctity of their own homes without being 
    subject to an illegal search and seizure of property. Today, we have a 
    different representation of ``property." Several amendments do imply the 
    right to privacy. The first amendment allows us the privacy to our own 
    beliefs. The third amendment grants us dominion and privacy in our own 
    homes. The fourth amendment allows us privacy over our person and property, 
    protecting us from unlawful search and seizure. Lastly, the most important 
    implied right comes from the fifth, that shields us from being compelled to 
    self-incriminate, or alternatively, gives us personal privacy over our 
    information. Today, a digital photo is property. An email is property. The 
    contents and data in an iPhone are property. If we allow for the 
    aforementioned liberties to be taken away from us, we invite more and more 
    picking of fruit from the poisoned tree. If we are to hold the stipulations 
    of the Bill of Rights in the same regard as we always have, then we have to 
    acknowledge these new definitions and acknowledge that these are also 
    protected by that same document. Hence, a supporting argument to value and 
    protect individual privacy is that the Bill of Rights we swear by agrees.
    
    \par One might argue that if we are to use the temporal context of the 
    Constitution in our favor, we must also acknowledge the analogous 
    competition: modern national security threats. With modern terrorism, 
    widespread access to semi-automatic weapons and explosives, and other 
    advances in the violent sciences, we are under the impression that our national 
    security is threatened from unaffiliated individuals with strong agendas, 
    now more than ever. We are under the impression that, with the Patriot Act 
    and other national security measures implemented since the 2001 wake-up 
    call, we are safe and the government has done its part. However, the 
    statistics disagree. ``Security theatre" is a concept that has come about to 
    describe the illusion of security that measures such as more invasive TSA 
    screenings provide. In reality, since 2004, the Government Accountability 
    Office has confirmed that 16 individuals accused of involvement in terrorist 
    plots have flown through US airports 23 times since 2004, completely 
    undetected by TSA behavior detection officers (Keteyian, 2010).
    Full-body scanners at TSA security checkpoints were supposed to protect the 
    nation's travelers from terror attacks, and perhaps they do deter attacks 
    and discourage would-be terrorists from attempting to bring weapons on 
    planes, but this comes at the cost of every single passenger's individual 
    privacy. On top of that, fear-mongering by the government has brainwashed 
    individuals into thinking that this is necessitous; that it is for the 
    greater good. Even if these measures genuinely protect our citizens, is it 
    right for the government to use fear as a tool to push their agenda?
    
    \par Perhaps the biggest offender in stripping the American populace of 
    their right to personal privacy is the NSA. From the NSA's website:
    ``In the past, we used our close partnership with the FBI to collect 
    bulk telephone records on an ongoing basis using a Top Secret order from the 
    Foreign Intelligence Surveillance Court (FISA). The metadata we collected 
    from this program gave us information about what communications you sent and 
    received, who you talked to, where you were when you talked to them, the 
    lengths of your conversations, and what kind of device you were using."
    Statistics report that the NSA collects and stores almost 200 million text 
    messages a day globally (Mardell, 2014). They store data about international 
    payments, social network communications, browsing activity, and more 
    (SPIEGEL, 2013). Perhaps worst of all, by Section 215 of the Patriot Act, 
    the government may compel companies to hand over information and create 
    backdoors to allow their surveillance to continue. The claim is that all of 
    this information is in pursuit of foreign intelligence and national 
    security, and that the rest of the data is expunged or held highly secure.  
    But what if the NSA makes a mistake, and all of this data is leaked or 
    released? It is definitely not outside the realm of possibility, considering 
    famous instances of internally sourced NSA leaks have happened before! With 
    the case of Edward Snowden, we saw just how much of our liberties were being 
    surreptitiously taken away from us, and based on the outlash from government 
    against Snowden and the ensuing manhunt, they certainly did not want to 
    reward such behavior: behavior that finally let the citizens of the United 
    States out of the dark, and exposed them to the vices of big data and the 
    true exploitation of modern technology and modern fear.

    \par Protection of our individual privacy does not have to be at the expense 
    national security. Invasive domestic surveillance, without probable cause or 
    reasonable suspicion, is unambiguously a violation of the rights granted to 
    us by the Constitution and its modern-day implications. With the cost of 
    data storage decreasing, and technological means of data collection 
    increasing, we cannot expect for the abusive collection of our data to slow 
    or stop without explicitly making it clear that we are aware of the 
    violation of our rights and refuse to let it persist. Through 
    whistleblowers, outspoken citizens, and well-intentioned businesses, we have 
    been able to make strides toward the protection of our data from unlawful 
    pervasion by government officials. National security should not come at the 
    cost of personal liberty, and so long as we allow the fear-mongering to 
    continue and allow ourselves to be deluded by talks that it is all ``for the 
    greater good," we will never be able to have control or security over the 
    personal property that is most near and dear to us today: our information.  
    However, the question of whether or not it is all worth it, or for the 
    greater good, is contestable on both sides. Until the matter has been 
    resolved, and a consensus that balances morality, individual privacy, and 
    national security can be reached, we should not let it continue. 


    %END BODY OF PAPER
    \begin{workscited}
        \bibent
        Mardell, Mark. ``Report: NSA 'collected 200m texts per day'" \textit{BBC 
        News}. BBC, 17 January 2014.

        \bibent
        SPIEGEL. ``NSA Spies on International Payments" SPIEGEL.de. 15 September 
        2013.

       \bibent
       UMD.edu, ``START: American Deaths in Terrorist Attacks" \textit{UMD.edu} 
       October 2015.

       \bibent
       Keteyian, Armen. ``TSA's Program to Spot Terrorists a \$200M Sham?" 
       \textit{CBS News}. CBS Networks, 19 May 2010.

       \bibent
       Office of the Press Secretary, ``Statement by the President on the 
       Shootings at Umpqua Community College, Roseburg, Oregon". 
       \textit{Archives.gov} ObamaWhiteHouse. 1 October 2015.
    
    \end{workscited}
\end{flushleft}
\end{document}
